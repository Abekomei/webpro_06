\documentclass[uplatex,dvipdfmx]{jsarticle}

% 画像、URL、コード表示、表作成に必要なパッケージ
\usepackage[dvipdfmx]{graphicx}
\usepackage{url}
\usepackage{float}
\usepackage{listings}
\usepackage{color}
\usepackage{geometry}

% 余白の調整(表や画像がはみ出ないように)
\geometry{left=25mm, right=25mm, top=30mm, bottom=30mm}

% コード表示の設定
\lstset{
  basicstyle={\ttfamily},
  identifierstyle={\small},
  commentstyle={\small\itshape},
  keywordstyle={\small\bfseries},
  ndkeywordstyle={\small},
  stringstyle={\small\ttfamily},
  frame={tb},
  breaklines=true,
  columns=[l]{fullflexible},
  numbers=left,
  xrightmargin=0zw,
  xleftmargin=3zw,
  numberstyle={\scriptsize},
  stepnumber=1,
  numbersep=1zw,
  lineskip=-0.5ex,
  keepspaces=true
}

% タイトル情報
\title{Web アプリケーション仕様書}
\author{千葉工業大学 情報工学科 1年\\学籍番号 25G1006\\氏名 安部 孔明}
\date{\today}

\begin{document}

\maketitle
\tableofcontents
\newpage

% ==================================================
% 第1章:ソースコードの管理場所(先生の指示により先頭へ配置)
% ==================================================
\section{ソースコードの管理場所}
本課題にて開発したアプリケーションのソースコードは,以下のGitHubリポジトリにて公開・管理している.

\url{https://github.com/Abekomei/webpro_06}

% ==================================================
% 第2章:はじめに
% ==================================================
\section{はじめに}
本レポートは,Webプログラミングの最終課題として開発した「統合データ管理システム」の仕様書である.

本システムは,日常生活における異なる属性のデータ(ゲームのキャラ評価,カラオケの楽曲,サブスクリプション契約)を,統一されたWebインターフェース上で一元管理することを目的としている.

本ドキュメントは,以下の3部構成で記述する.
\begin{enumerate}
    \item 利用者向けマニュアル
    \item 管理者向けマニュアル
    \item 開発者向け仕様書
\end{enumerate}

\newpage

% ==================================================
% 第3章:利用者向けマニュアル
% ==================================================
\section{利用者向けマニュアル}

\subsection{システム概要とトップページ}
本システムは,トップページ(メニュー画面)から各アプリケーションへアクセスし,データの閲覧・追加・削除を行うことができる.

\subsubsection{アクセス方法}
Webブラウザ(Google Chrome等)より以下のURLにアクセスする.
\begin{center}
\url{http://localhost:8081/}
\end{center}

アクセスすると以下のメニュー画面が表示され,目的の管理アプリを選択できる.

\begin{figure}[H]
  \centering
  \includegraphics[width=12cm]{top.png}
  \caption{トップページ(メニュー画面)}
\end{figure}

\subsection{各機能の操作}
ここでは「Apex Legends キャラ管理」を例に操作方法を説明する(他アプリも操作は共通である).

\subsubsection{一覧画面}
各アプリのメイン画面では,登録データが表形式で表示される.
\begin{itemize}
    \item \textbf{詳細}: データの全項目を確認できる詳細画面へ移動する.
    \item \textbf{編集}: 登録済みのデータを修正する画面へ移動する.
    \item \textbf{削除}: データをリストから削除する(確認画面なしで即時実行される).
    \item \textbf{+新規追加}: 新しいデータを登録するフォームへ移動する.
\end{itemize}

\begin{figure}[H]
  \centering
  \includegraphics[width=12cm]{apex_list.png}
  \caption{一覧画面の例(Apex Legends)}
\end{figure}

\subsubsection{詳細画面}
一覧画面で「詳細」をクリックすると,そのデータに関するすべての情報(メモなど)が表示される.

\begin{figure}[H]
  \centering
  \includegraphics[width=10cm]{apex_detail.png}
  \caption{詳細画面の例(Apex Legends)}
\end{figure}

\newpage

% ==================================================
% 第4章:管理者向けマニュアル
% ==================================================
\section{管理者向けマニュアル}

\subsection{システム要件}
\begin{itemize}
    \item \textbf{OS}: Windows,macOS,Linux
    \item \textbf{実行環境}: Node.js (v14.x以上推奨)
    \item \textbf{依存ライブラリ}: Express,EJS
\end{itemize}

\subsection{起動手順}
ターミナルでプロジェクトディレクトリに移動し,以下のコマンドを実行する.

\begin{lstlisting}[language=bash, caption=サーバー起動コマンド]
$ node apph.js
\end{lstlisting}

\texttt{Server started on port 8081!} と表示されれば起動完了である.

\subsection{データの仕様(注意点)}
本システムは学習用プロトタイプのため,データベースを使用せずメモリ上(変数)でデータを管理している.\textbf{サーバーを再起動すると,追加・削除されたデータは初期状態に戻る}仕様となっている.

\newpage

% ==================================================
% 第5章:開発者向け仕様書(詳細仕様)
% ==================================================
\section{開発者向け仕様書}
本章では,システムの内部構造,API仕様,および画面遷移について記述する.

\subsection{共通仕様}
全システム共通で,以下の技術スタックを採用している.
\begin{itemize}
    \item フレームワーク: Express
    \item テンプレートエンジン: EJS
    \item データ管理: 配列変数(インメモリ)
\end{itemize}

\subsection{ディレクトリ構成}
\begin{verbatim}
webpro_06/
  ├── apph.js            (コントローラ兼モデル)
  ├── views/
  │    ├── top.ejs           (トップページ)
  │    ├── common_list.ejs   (共通一覧テンプレート)
  │    └── common_detail.ejs (共通詳細テンプレート)
  ├── public/            (静的ファイル:フォーム)
  │    ├── apex_new.html
  │    ├── karaoke_new.html
  │    └── subscription_new.html
  └── package.json
\end{verbatim}

\newpage

\subsection{機能別 APIおよび画面遷移仕様}

各アプリケーションは共通の画面遷移ロジックに基づいて設計されている.詳細な遷移フローを以下の図に示す.

\begin{figure}[H]
  \centering
  \includegraphics[width=14cm]{a.png}
  \caption{システム共通の画面遷移図}
\end{figure}

以下の表に,各アプリケーションのデータ構造およびAPI仕様を示す.

% --- トップページ ---
\subsubsection{トップページ (メニュー)}
システム全体の入り口となるランディングページである.

\begin{table}[H]
\caption{トップページ API仕様}
\begin{center}
\begin{tabular}{|l|l|l|l|} \hline
\textbf{機能} & \textbf{メソッド} & \textbf{パス} & \textbf{処理内容} \\ \hline
メニュー表示 & GET & / & top.ejs をレンダリングし,リンクを表示 \\ \hline
\end{tabular}
\end{center}
\end{table}

% --- APEX ---
\subsubsection{Apex Legends キャラ管理}
ゲームキャラクターのTier(強さランク)を管理する機能.

\textbf{データ構造(変数名: apexData)}
\begin{table}[H]
\caption{Apex データ構造}
\begin{center}
\begin{tabular}{|l|l|l|} \hline
\textbf{プロパティ名} & \textbf{型} & \textbf{説明} \\ \hline
id & Number & データ識別用ID \\ \hline
name & String & キャラクター名 \\ \hline
tier & String & 強さのランク(S, Aなど) \\ \hline
memo & String & 詳細メモ \\ \hline
\end{tabular}
\end{center}
\end{table}

\textbf{API仕様}
\begin{table}[H]
\caption{Apex API仕様}
\begin{center}
\begin{tabular}{|l|l|l|l|} \hline
\textbf{機能} & \textbf{メソッド} & \textbf{パス} & \textbf{処理内容} \\ \hline
一覧表示 & GET & /apex & データを一覧表示 (common\_list.ejs) \\ \hline
詳細表示 & GET & /apex/detail/:id & 指定IDの詳細を表示 (common\_detail.ejs) \\ \hline
新規登録画面 & GET & /apex/create & 入力フォーム (apex\_new.html) へリダイレクト \\ \hline
データ追加 & POST & /apex/add & フォーム値を受け取り配列に追加し一覧へ遷移 \\ \hline
編集画面 & GET & /apex/edit/:id & 編集フォームを表示 \\ \hline
データ更新 & POST & /apex/update/:id & データを更新し一覧へ遷移 \\ \hline
データ削除 & GET & /apex/delete/:id & 指定IDを配列から削除し一覧へ遷移 \\ \hline
\end{tabular}
\end{center}
\end{table}

% --- カラオケ ---
\subsubsection{カラオケ楽曲管理}
練習したい曲やキー設定を管理する機能.

\begin{figure}[H]
  \begin{minipage}{0.48\textwidth}
    \centering
    \includegraphics[width=7.5cm]{karaoke_list.png}
    \caption{カラオケ 一覧}
  \end{minipage}
  \hfill
  \begin{minipage}{0.48\textwidth}
    \centering
    \includegraphics[width=7.5cm]{karaoke_detail.png}
    \caption{カラオケ 詳細}
  \end{minipage}
\end{figure}

\textbf{データ構造(変数名: karaokeData)}
\begin{table}[H]
\caption{カラオケ データ構造}
\begin{center}
\begin{tabular}{|l|l|l|} \hline
\textbf{プロパティ名} & \textbf{型} & \textbf{説明} \\ \hline
id & Number & データ識別用ID \\ \hline
song & String & 曲名 \\ \hline
artist & String & アーティスト名 \\ \hline
key & String & キー設定(原曲, +2など) \\ \hline
memo & String & メモ(サビが高いなど) \\ \hline
\end{tabular}
\end{center}
\end{table}

\textbf{API仕様}
\begin{table}[H]
\caption{カラオケ API仕様}
\begin{center}
\begin{tabular}{|l|l|l|l|} \hline
\textbf{機能} & \textbf{メソッド} & \textbf{パス} & \textbf{処理内容} \\ \hline
一覧表示 & GET & /karaoke & データを一覧表示 (common\_list.ejs) \\ \hline
詳細表示 & GET & /karaoke/detail/:id & 指定IDの詳細を表示 (common\_detail.ejs) \\ \hline
新規登録画面 & GET & /karaoke/create & 入力フォーム (karaoke\_new.html) へリダイレクト \\ \hline
データ追加 & POST & /karaoke/add & フォーム値を受け取り配列に追加し一覧へ遷移 \\ \hline
編集画面 & GET & /karaoke/edit/:id & 編集フォームを表示 \\ \hline
データ更新 & POST & /karaoke/update/:id & データを更新し一覧へ遷移 \\ \hline
データ削除 & GET & /karaoke/delete/:id & 指定IDを配列から削除し一覧へ遷移 \\ \hline
\end{tabular}
\end{center}
\end{table}

% --- サブスク ---
\subsubsection{サブスクリプション管理}
契約中のサービスと月額料金を管理する機能.一覧表示時に合計金額を計算するロジックを含む.

\begin{figure}[H]
  \begin{minipage}{0.48\textwidth}
    \centering
    \includegraphics[width=7.5cm]{sub_list.png}
    \caption{サブスク 一覧}
  \end{minipage}
  \hfill
  \begin{minipage}{0.48\textwidth}
    \centering
    \includegraphics[width=7.5cm]{sub_detail.png}
    \caption{サブスク 詳細}
  \end{minipage}
\end{figure}

\textbf{データ構造(変数名: subData)}
\begin{table}[H]
\caption{サブスク データ構造}
\begin{center}
\begin{tabular}{|l|l|l|} \hline
\textbf{プロパティ名} & \textbf{型} & \textbf{説明} \\ \hline
id & Number & データ識別用ID \\ \hline
service & String & サービス名 \\ \hline
price & Number & 月額料金(円) \\ \hline
renewal & String & 更新日 \\ \hline
memo & String & メモ \\ \hline
\end{tabular}
\end{center}
\end{table}

\textbf{API仕様}
\begin{table}[H]
\caption{サブスク API仕様}
\begin{center}
\begin{tabular}{|l|l|l|l|} \hline
\textbf{機能} & \textbf{メソッド} & \textbf{パス} & \textbf{処理内容} \\ \hline
一覧表示 & GET & /sub & 合計金額を計算し一覧表示 (common\_list.ejs) \\ \hline
詳細表示 & GET & /sub/detail/:id & 指定IDの詳細を表示 (common\_detail.ejs) \\ \hline
新規登録画面 & GET & /sub/create & 入力フォーム (subscription\_new.html) へリダイレクト \\ \hline
データ追加 & POST & /sub/add & フォーム値を受け取り配列に追加し一覧へ遷移 \\ \hline
編集画面 & GET & /sub/edit/:id & 編集フォームを表示 \\ \hline
データ更新 & POST & /sub/update/:id & データを更新し一覧へ遷移 \\ \hline
データ削除 & GET & /sub/delete/:id & 指定IDを配列から削除し一覧へ遷移 \\ \hline
\end{tabular}
\end{center}
\end{table}

\newpage

\subsection{実装の工夫点(テンプレートの共通化)}
本システムでは,開発効率と保守性を高めるため,3つの異なるアプリに対し\textbf{共通のEJSテンプレート}を使用している.

\subsubsection{Configオブジェクトによる制御}
\texttt{apph.js} 側で,各アプリの表示項目(カラム名)をオブジェクトとして定義している.

\begin{lstlisting}[language=Java, caption={設定オブジェクトの例(Apex用)}]
const APEX_CONFIG = {
    title: "APEX キャラTier表",
    baseUrl: "/apex",
    listColumns: [
        { label: "ID", key: "id" },
        { label: "キャラ名", key: "name" },
        { label: "Tier", key: "tier" }
    ],
    // ...
};
\end{lstlisting}

\subsubsection{共通テンプレートの利用}
定義したConfigをテンプレートに渡すことで,単一のファイルで多様なデータを表示可能とした.

\begin{lstlisting}[language=html, caption={common\_list.ejs(抜粋)}]
<% for (let col of columns) { %>
    <th><%= col.label %></th>
<% } %>
\end{lstlisting}

これにより,新しいアプリを追加する際もHTMLを作成する必要がなく,設定の追加のみで実装が可能となっている.

\end{document}